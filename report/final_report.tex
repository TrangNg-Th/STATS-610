\documentclass[11pt, letterpaper]{article}

% =============================================================================
% Packages
% =============================================================================
\usepackage[utf8]{inputenc}
\usepackage[T1]{fontenc}
\usepackage{amsmath, amssymb, amsthm}
\usepackage{graphicx}
\usepackage{booktabs}
\usepackage[margin=1in]{geometry}
\usepackage{hyperref}
\usepackage{natbib}
\usepackage{setspace}
\usepackage{float}

% Set line spacing
\onehalfspacing

% =============================================================================
% Title
% =============================================================================
\title{STATS 610 Final Project:\\
Replication of [Paper Title]}
\author{[Student Name]}
\date{\today}

\begin{document}

\maketitle

% =============================================================================
% Abstract
% =============================================================================
\begin{abstract}
This report presents a replication of key statistical results from [Paper Citation]. Using the supplementary data and methods described in the original paper, we reproduce the main findings and provide a critical analysis of the methodology. Our replication confirms/extends [brief summary of findings].
\end{abstract}

% =============================================================================
% 1. Introduction
% =============================================================================
\section{Introduction}

[Introduce the paper being replicated. Explain the research question, why this paper was selected, and the importance of the statistical methods used.]

The original paper by [Author(s)] investigates [research question]. The key contributions include:
\begin{itemize}
    \item [Contribution 1]
    \item [Contribution 2]
    \item [Contribution 3]
\end{itemize}

% =============================================================================
% 2. Data Description
% =============================================================================
\section{Data Description}

[Describe the data from the paper's supplementary materials.]

\subsection{Data Source}

The data for this analysis comes from [source description]. The dataset includes $n = [N]$ observations with the following variables:

\begin{itemize}
    \item \textbf{Variable 1}: [Description]
    \item \textbf{Variable 2}: [Description]
    \item \textbf{Variable 3}: [Description]
\end{itemize}

\subsection{Summary Statistics}

% Add table with summary statistics
% \begin{table}[H]
% \centering
% \caption{Summary Statistics}
% \begin{tabular}{lcccc}
% \toprule
% Variable & Mean & SD & Min & Max \\
% \midrule
% Variable 1 & & & & \\
% Variable 2 & & & & \\
% \bottomrule
% \end{tabular}
% \end{table}

% =============================================================================
% 3. Methods
% =============================================================================
\section{Methods}

[Describe the statistical methods used in the original paper that you are replicating.]

\subsection{Statistical Approach}

The original paper employs [method name] to address [research question]. The key assumptions include:
\begin{enumerate}
    \item [Assumption 1]
    \item [Assumption 2]
\end{enumerate}

\subsection{Model Specification}

% Add mathematical model specification
% \begin{equation}
% y_i = \beta_0 + \beta_1 x_{1i} + \epsilon_i, \quad \epsilon_i \sim N(0, \sigma^2)
% \end{equation}

% =============================================================================
% 4. Results
% =============================================================================
\section{Results}

[Present the replicated results and compare them with the original paper's findings.]

\subsection{Main Findings}

% Add figure from R analysis
% \begin{figure}[H]
% \centering
% \includegraphics[width=0.8\textwidth]{../figures/figure1.png}
% \caption{[Figure description]}
% \label{fig:main}
% \end{figure}

\subsection{Comparison with Original Results}

% Add comparison table
% \begin{table}[H]
% \centering
% \caption{Comparison of Results}
% \begin{tabular}{lcc}
% \toprule
% Parameter & Original Paper & Our Replication \\
% \midrule
% $\beta_1$ & & \\
% $\beta_2$ & & \\
% \bottomrule
% \end{tabular}
% \end{table}

% =============================================================================
% 5. Discussion
% =============================================================================
\section{Discussion}

[Discuss the implications of your replication results. Address any discrepancies with the original findings and possible explanations.]

\subsection{Interpretation}

[Interpret the replicated results in the context of the original research question.]

\subsection{Limitations}

[Discuss any limitations of the replication effort.]

% =============================================================================
% 6. Conclusion
% =============================================================================
\section{Conclusion}

[Summarize the replication effort and main conclusions.]

This replication study has [confirmed/extended/challenged] the original findings of [Author(s)]. The key takeaways include:
\begin{itemize}
    \item [Takeaway 1]
    \item [Takeaway 2]
\end{itemize}

% =============================================================================
% References
% =============================================================================
\bibliographystyle{apalike}
% \bibliography{references}  % Uncomment when you have a references.bib file

\begin{thebibliography}{99}
\bibitem{original} [Author(s)]. [Year]. [Paper Title]. \textit{[Journal Name]}, Volume(Issue), pages.
\end{thebibliography}

\end{document}